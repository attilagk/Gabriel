\documentclass[letterpaper]{article}
\usepackage{geometry, graphicx}
\usepackage{natbib}
\bibliographystyle{plainnat}

\begin{document}

\title{Sphingolipid and phospholipid level changes in human brain indicate
defects in lipid processing in VPS13A-disease
	\\
{\large Supplementary Methods and Figures}}

\author{Gabriel Miltenberger-Miltenyi, Attila Jones, Amber M.~Tetlow,  Vasco A.~Conceição, 
\\
	John F.~Crary, Ricky Michael Ditzel Jr, Kurt Farrell, Renu Nandakumar,
\\
	Brandon Barton, Barbara I.~Karp  Alana Kirkby, Debra J.~Lett,
\\
	Karin Mente, Susan Morgello, David K.~Simon, Ruth H.~Walker}
\date{}
\maketitle

%\renewcommand*\contentname{Supplementary Methods}
\renewcommand{\contentsname}{Supplementary Methods}
\tableofcontents
\renewcommand*\listfigurename{Supplementary Figures}
\listoffigures

%\section{Supplementary Methods}

\section{Lipidomics}

The 593 lipid species are classified into 34 lipid groups. 
Lipid extracts were prepared from tissue homogenates spiked with multiple
class-based internal standards using a modified Bligh and Dyer method, and
they were analyzed on a platform comprising of Agilent 1260 Infinity HPLC
integrated to Agilent 6490A QQQ mass spectrometer controlled by Masshunter v
7.0 (Agilent Technologies, Santa Clara, CA). Glycerophospholipids and
sphingolipids were separated with normal-phase HPLC as described previously25,
with a few modifications. An Agilent Zorbax Rx-Sil column (2.1 x 100 mm, 1.8
$\mu$m) maintained at $25\circ$C was used under the following conditions: mobile phase A
(chloroform: methanol: ammonium hydroxide, 89.9:10:0.1, v/v) and mobile phase
B (chloroform: methanol: water: ammonium hydroxide, 55:39:5.9:0.1, v/v); 95\% A
for 2 min, decreased linearly to 30\% A over 18 min and further decreased to
25\% A over 3 min, before returning to 95\% A over 2 min and held for 6 min.
Separation of sterols and glycerolipids was carried out on a reverse phase
Agilent Zorbax Eclipse XDB-C18 column (4.6 x 100 mm, 3.5um) using an isocratic
mobile phase, chloroform, methanol, and 0.1 M ammonium acetate (25:25:1) at a
flow rate of 300 $\mu$l/min.
Quantification of lipid species was accomplished using multiple reaction
monitoring (MRM) transitions25 under both positive and negative ionization
modes in conjunction with referencing of class-based internal standards: PA
14:0/14:0, PC 14:0/14:0, PE 14:0/14:0, phosphatidylglycerol (PG) 15:0/15:0,
phosphatidylinositol (PI)17:0-20:4; phosphatidylserine (PS) 14:0/14:0, BMP
14:0/14:0, Hemi BMP 14:0; LPC 13:0,  LPE 14:0, LPI 13:0, Cer d18:1/17:0, SM
d18:1/12:0, dhSM d18:0/12:0, GalCer d18:1/12:0, sulfatide (Sulf) d18:1/12:0,
LacCer d18:1/12:0, D7-cholesterol, CE 17:0, MG 17:0, DG 28:0/14:0 D5-TG
16:0/18:0/16:0 (Avanti Polar Lipids, Alabaster, AL). 

Multiple class-based internal standards of similar ionization efficiencies
were selected and spiked at an appropriate concentration
(0.05-156.14pmol/$\mu$l) as published elsewhere~\citep{Chan2017}. Each
internal standard was later used to calculate the concentrations of
corresponding lipid classes by first calculating ratio between measured
intensities of a lipid species and that of corresponding internal standard
multiplied by the known concentration of the internal standard.

Relative molar amounts of lipid species in each sample were calculated based
on appropriate class-based internal standards and normalized against tissue
wet weight (nmol/ug). The data were then normalized to the sum of these molar
contributions to obtain comparable relative contributions of each lipid
species or class and expressed as mol\% relative to total amount of all
measured lipid classes~\citep{Hartler2014,Guan2013}.

Lipid abbreviations are as follows:
\begin{description}
\item[FC] Free Cholesterol
\item[CE] Cholesterol Ester
\item[AC] Acyl Carnitine
\item[MG] Monoacylglycerol
\item[DG] Diacylglycerol
\item[TG] Triacylglycerol
\item[Cer] Ceramide
\item[dhCer] Dihydroceramide
\item[SM] Sphingomyelin
\item[dhSM] Dihydrosphingomyelin
\item[Sulf] Sulfatide
\item[MHCer] Monohexosylceramide (galactosylceramide + glucosylceramide)
\item[LacCer] Lactosylceramide
\item[GM3] Monosialodihexosylganglioside
\item[GB3] Globotriaosylceramide
\item[PA] Phosphatidic acid
\item[PC] Phosphatylcholine
\item[PCe] Ether phosphatidylcholine
\item[PE] Phosphatidylethanolamine
\item[PEp] Plasmalogen phosphatidylethanolamine
\item[PS] Phosphatidylserine
\item[PI] Phosphatidylinositol
\item[PG] Phosphatidylglycerol
\item[BMP] Bis(monoacylglycero)phosphate
\item[AcylPG] Acyl Phosphatidylglycerol
\item[LPC] Lysophosphatidylcholine
\item[LPCe] Ether lysophosphatidylcholine
\item[LPE] Lysophosphatidylethanolamine
\item[LPEp] Plasmogen Lysophosphatidylethanolamine
\item[LPI] Lysophosphatidylinositol
\item[LPS] Lysophosphatidylserine
\item[NAPE] N-Acyl Phosphatidylethanolamine
\item[NAPS] N-Acyl Phosphatidylserine
\item[NSer] N-Acyl Serine
\end{description}
\section{Data overview}

Our lipidomic assay quantified the level of 593 lipid species in three brain
regions (DLPFC, CN and Putamen) from each of $n=10$ subjects (6 control and 4
ChAc).  Besides ChAc status (Dx variable), we knew the age at each subject's
death (AgeAtDeath variable).

The 593 lipid species were classified into 34 lipid groups based on chemical
structure.  The smallest group (FC, free cholesterol group) had only one
species; for the rest of the groups, the number $n_g$ of lipid species ranged
from 6 (NSer, N-acyl serine group) to 42 (TG, tryacylglycerol group).

Given $3n = 30$ observations and 593 lipid species, we refer to the $30 \times
593$ matrix of lipid levels as \emph{species-level data} and we based our
final inferences on these data.  Given 34 lipid groups, we also produced a
smaller, $30 \times 34$ data matrix by summing the level of lipid species
within groups; we call this matrix \emph{group-level data} and we used it only
for preliminary analyses.

Note that in the species-level data we have $30 \times n_g$ observations on
each group $g$ but in the group-level data only $30$ observations on each
group.  The number of observations on each lipid species in the species-level
data is also only $30$.  Therefore, making inferences on lipid groups based on
the species-level data comes with the advantage of $n_g \times$ more
observations relative to making inferences (i) on groups based on the
group-level data or (ii) on species based on the species-level data.

According to the central limit theorem the $n_g \times$ more observations,
under certain conditions (see below), may afford $\approx \sqrt n_g \times$
reduction of error.  This is the most beneficial for large lipid groups.  For
instance, the $g = \mathrm{PC}$ (phosphatidylcholine) group contains $n_g =
25$ species, which means ideally $\approx 5 \times$ reduction in error.

The aforementioned advantage of $n_g \times$
more observation on lipid group $g$ based on the species-level data depends
entirely on the assumption that the level of species $c$ is conditionally
i.i.d across all $c \in g$ for a given ChAc status, brain region and age at
death.  We confirmed this assumption using heatmaps of a standardized  version
of the species-level data.  Standardization proved to be essential because
lipid level typically varies greatly across all species $c \in g$.

\section{Data standardization}

We standardized lipid levels for the sake of all visualizations (all heatmaps
and Fig.~\ref{fig:individual-changes}-\ref{fig:individual-changes-simple}) and
all model-based
statistical inference.  In the species-level data, for each lipid species $c$,
we standardized lipid levels $z_{ci}$ across the $i = 1...30$ samples to
obtain the standardized levels $\{y_{ci}\;:\; i = 1...30\}$:
\begin{equation}
	y_{ci} = (z_{ci} - \bar{z}_c) / s_c,
\end{equation}

where $\bar{z}_c = 30^{-1} \sum_{i=1}^{30} z_{ci}$ is the average level of
species $c$ and $s^2_c = (30 - 1)^{-1} \sum_{i=1}^{30} (z_{ci} - \bar{z}_c)^2$
is the corresponding sample variance.  After standardization the unit of lipid
level change is the sample standard deviation $s_c$.

We performed similar standardization also on the group-level data.

\section{Regression model}

For each lipid group $g$ we modeled standardized lipid level using linear
regression with the symbolic formula `Level $\sim$ 1 + Region + Dx:Region +
AgeAtDeath`.  This means that we regressed standardized lipid level (Level) on
a linear combination of fixed effects terms: brain region (Region), the
combination of region and ChAc status (Dx:Region, the interaction term of our
main interest), and on the age at death (AgeAtDeath).  In addition, we added a
random effects term that modeled subject-to-subject heterogeneity of lipid
levels.

Therefore our model could be written as
\begin{equation}
	\mathbf{y}_{g} = \sum_{j=0}^{6} \beta_{gj} \mathbf{x}_{gj} + \mathbf{\gamma}_{g} + \mathbf{\epsilon}_{g},
	\label{eq:main-model}
\end{equation}
where $\mathbf{y}_{g}$ is the $m$-length data vector of standardized lipid
level from $m$ observations, where $m = 30 n_g$ for the species-level data and
$m = 30$ for the group-level data; $\{\mathbf{x}_{gj}\}_j$ are $m$-length
data vectors as defined in the Table~\ref{tab:data-vectors};
$\mathbf{\gamma}_{g}$ and $\mathbf{\epsilon}_{g}$ are both random $m$-vectors
such that
$\mathbf{\gamma}_{g} \sim \mathcal{N}(\mathbf{0}_m, \sigma^2_{\gamma
g}\Omega_g)$ and
$\mathbf{\epsilon}_{g} \sim \mathcal{N}(\mathbf{0}_m, \sigma^2_{\epsilon g}
\mathbf{I}_m)$, respectively.  Here $\Omega$ is the $m \times m$ correlation
matrix of $\mathbf{\gamma}_{g}$ constrained such that all observations from
the same subject change by the same random amount (they have a correlation of
one between each other), while observations from different subjects are
independent from each other (they have zero correlation).

\begin{table}
\begin{tabular}[]{lll}
 \hline
 symbol            &  alt.~symbol                        & description \\
 \hline
 $\mathbf{y}_{g}$  & $\mathbf{y}_{g}$                    & standardized lipid level in group $g$ \\
 $\mathbf{x}_{g1}$ & $\mathbf{x}_{g\mathrm{CN}}$         & 1 for samples from CN, 0 otherwise \\
 $\mathbf{x}_{g2}$ & $\mathbf{x}_{g\mathrm{Putamen}}$    & 1 for samples from Putamen, 0 otherwise \\
                   & $\mathbf{x}_{g\mathrm{DLPFC}}$      & 1 for samples from DLPFC, 0 otherwise \\
                   & $\mathbf{x}_{g\mathrm{ChAc}}$       & 1 for samples from ChAc subjects, 0 otherwise \\
 $\mathbf{x}_{g6}$ & $\mathbf{x}_{g\mathrm{AgeAtDeath}}$ & age at death in years \\
 \hline
\end{tabular}
\caption{Data vectors used in the model (Eq.~\ref{eq:main-model}).
$\mathbf{x}_{g3}, \mathbf{x}_{g4}, \mathbf{x}_{g5}$ are part of the Dx:Region
interaction term and defined as the element-wise vector product $\mathbf{x}_{g\mathrm{ChAc}} \circ
\mathbf{x}_{gr}$, where region $r =$ DLPFC, CN and Putamen,
respectively. The length of all vectors is $30 n_g$ for the species-level data
and $30$ for the group-level data, where $n_g$ is the number of lipid species
in group $g$.}
\label{tab:data-vectors}
\end{table}

The regression parameters $\{\beta_{gj}\}_j$ and their interpretations are
listed in Table~\ref{tab:parameters}.  The parameters of our main interest are
$\beta_{3g}, \beta_{4g}, \beta_{5g}$ corresponding to the terms
$\{$Dx[T.ChAc]:Region[$r$]$ \; : \; r \in \{\mathrm{DLPFC}, \mathrm{CN},
\mathrm{Putamen}\}\}$, respectively.

\begin{table}
	\small
\begin{tabular}{lp{2.5cm}p{2.5cm}p{6cm}}
	 \hline
	 param.              & term                      & interpretation of parameter's effect           & interpretation of term \\
	 \hline
	 \hline
	 $\beta_{0g}$        & Intercept                 & reference level                                & avg. standardized lipid level in DLPFC across all control subjects and all lipid species within lipid group \\
	 \hline
	 $\beta_{1g}$        & Region[T.CN]              & fixed effect of the CN region w.r.t DLPFC      & avg. standardized lipid level in CN w.r.t DLPFC across all control subjects and lipid species within lipid group \\
	 \hline
	 $\beta_{2g}$        & Region[T.Putamen]         & fixed effect of the Putamen region w.r.t DLPFC & avg. standardized lipid level in Putamen w.r.t DLPFC across all control subjects and all lipid species within lipid group \\
	 \hline
	 $\beta_{3g}$        & Dx[T.ChAc]: Region[DLPFC]  & fixed, DLPFC-specific, effect of ChAc status   & avg. standardized lipid level in ChAc w.r.t Control, in DLPFC, across all ChAc subjects and all lipid species within lipid group \\
	 \hline
	 $\beta_{4g}$        & Dx[T.ChAc]: Region[CN]     & fixed, CN-specific, effect of ChAc status      & avg. standardized lipid level in ChAc w.r.t Control, in CN, across all ChAc subjects and all lipid species within lipid group \\
	 \hline
	 $\beta_{5g}$        & Dx[T.ChAc]: Region[Putamen]& fixed, Putamen-specific effect of ChAc status  & avg. standardized lipid level in ChAc w.r.t Control, in Putamen, across all ChAc subjects and all lipid species within lipid group \\
	 \hline
	 $\beta_{6g}$        &  AgeAtDeath               & fixed effect of age                            & change in avg.  standardized lipid level per year across all brain regions and all lipid species within lipid group \\
	 \hline
	 $\sigma_{\gamma g}$ & Subject Var               & random effect of cross-subject variation       & cross-subject standard deviation of avg. standardized lipid level across all lipid species within lipid group  \\
	 \hline
\end{tabular}
\caption{Model parameters, corresponding terms and interpretations.  Our main
	interest concerns $\beta_{3g}, \beta_{4g}$, and $\beta_{5g}$, corresponding
	to fixed, region specific, effects of ChAc for the DLPFC, CN and Putamen
	regions, respectively.
}
\label{tab:parameters}
\end{table}

We also fitted a more complex model that contains, as covariate, the
post-mortem interval in hours (PMI); with symbolic formula the model is
written as `Level $\sim$ 1 + Region + Dx:Region + AgeAtDeath + PMI`.  We found
that PMI had non-significant effect ($\alpha=0.01$) on lipid levels
(Fig.~S\ref{fig:PMI-pvals}) and the key results were little affected by the
correction for PMI
(Fig.~S\ref{fig:PMI-effect-on-estimates},~S\ref{fig:pvalues-corrected-wo-PMI}).
Therefore, in the main text we present results based on the simpler model that
lacks PMI correction.

\section{Model fitting}

We fitted models with the statsmodels Python package.  Goodness of fit was
confirmed using the Jarque-Bera test and normal Q-Q plots of
residuals.

\section{Hypothesis testing}

For each lipid group $g$ we formulated our inference on ChAc-associated lipid
changes using estimates $\hat{\beta}_{3g}, \hat{\beta}_{4g},
\hat{\beta}_{5g}$, which quantify the change of lipid level in ChAc w.r.t
Control in the DLPFC, CN and Putamen regions, respectively.

We also tested each of the following three null hypotheses: $\beta_{3g} = 0$,
$\beta_{4g} = 0$ and $\beta_{5g} = 0$ meaning no change in lipid group $g$'s
level in ChAc w.r.t Control in the DLPFC, CN and Putamen, respectively.
Testing for all 34 lipid groups resulted in $34 \times 3 = 102$ p-values.
We performed multiple hypothesis test correction conservatively taking all 102
p-values into account.  We performed both the Bonferroni and the
Benjamini-Hochberg procedures, and report in Results, for any given region,
those lipid groups for which the null hypothesis was rejected by the
Benjamini-Hochberg procedure.

\cleardoublepage
\phantomsection
\addcontentsline{toc}{section}{References}

\bibliography{supplementary}

\pagebreak
%\section{Supplementary Figures}

\setcounter{figure}{0}
\makeatletter 
\renewcommand{\figurename}{Supplementary Figure} % nice
\makeatother

\begin{figure}[p]
	\includegraphics[width=0.5\textwidth]{../../notebooks/2022-09-04-mixed-models/named-figure/raw-BMP-levels.png}
	\includegraphics[width=0.5\textwidth]{../../notebooks/2022-09-04-mixed-models/named-figure/norm-BMP-levels.png}
	\caption[Non-standardized and standardized BMP levels]{
	Non-standardized (left) and standardized (right) BMP levels.  Both
	non-standardized and standardized data have been normalized to mol \%.  In
	this figure and in Fig.~S\ref{fig:heatmap-Sulf}-S\ref{fig:heatmap-LPS} the
	darkest blue and red denote -3 and 3 $\times$ the standard deviation of
	lipid level across all 30 observations, respectively.
}
\label{fig:heatmap-BMP}
\end{figure}

\begin{figure}[p]
	\includegraphics[width=0.5\textwidth]{../../notebooks/2022-09-04-mixed-models/named-figure/raw-Sulf-levels.png}
	\includegraphics[width=0.5\textwidth]{../../notebooks/2022-09-04-mixed-models/named-figure/norm-Sulf-levels.png}
	\caption[Raw and standardized Sulf levels]{
	Non-standardized (left) and standardized (right) Sulf levels.  Both
	non-standardized and standardized data have been normalized to mol \%.  
}
\label{fig:heatmap-Sulf}
\end{figure}

\begin{figure}[p]
	\includegraphics[width=0.5\textwidth]{../../notebooks/2022-09-04-mixed-models/named-figure/raw-PCe-levels.png}
	\includegraphics[width=0.5\textwidth]{../../notebooks/2022-09-04-mixed-models/named-figure/norm-PCe-levels.png}
	\caption[Raw and standardized PCe levels]{
	Non-standardized (left) and standardized (right) PCe levels.  Both
	non-standardized and standardized data have been normalized to mol \%.  
}
\label{fig:heatmap-PCe}
\end{figure}

\begin{figure}[p]
	\includegraphics[width=0.5\textwidth]{../../notebooks/2022-09-04-mixed-models/named-figure/raw-LPS-levels.png}
	\includegraphics[width=0.5\textwidth]{../../notebooks/2022-09-04-mixed-models/named-figure/norm-LPS-levels.png}
	\caption[Raw and standardized LPS levels]{
	Non-standardized (left) and standardized (right) LPS levels.  Both
	non-standardized and standardized data have been normalized to mol \%.  
}
\label{fig:heatmap-LPS}
\end{figure}

\begin{figure}[p]
	\includegraphics[width=1.0\textwidth]{../../notebooks/2022-09-02-lipid-ch-ac/named-figure/clustermap-std.png}
	\caption[Clustermap of standardized lipid levels]{
	Clustermap of standardized lipid levels.  Clustering was performed with the
	clustermap function of the seaborn Python package using the default
	``average'' method and the default ``euclidean'' metric.
}
\label{fig:clustermap}
\end{figure}

\begin{figure}[p]
	\includegraphics[width=1.0\textwidth]{../../notebooks/2022-09-04-mixed-models/named-figure/all-qq-plots-fixed-fx.pdf}
	\caption[Goodness of fit: normality of residuals]{
	Goodness of fit assessed by quantile-quantile (QQ) plot of Pearson
	residuals. X axes correspond to quantiles of the standard normal
	distribution, while Y axes to sample quantiles. Proximity of the curves to
	the red diagonal means a better fit. Green symbols represent a simpler
	version of the model that we used for inference (the random effect term for
	subjects was omitted leaving only fixed effects in the model). Blue symbols
	refer to the same simpler model but with the model fitted to a group-level
	data, in which all lipid species within a group were summed for each sample.
}
\label{fig:QQ-plots}
\end{figure}

\begin{figure}[p]
	A

	\includegraphics[width=1.0\textwidth]{../../notebooks/2022-09-04-mixed-models/named-figure/t-value-heatmap-fixed-mixed-models.pdf}

	B

	\includegraphics[width=1.0\textwidth]{../../notebooks/2022-09-04-mixed-models/named-figure/p-value-heatmap-fixed-mixed-models.pdf}
	\caption[Comparing models and datasets in terms of $t$-statistics and
	$p$-values]{
	$t$-statistics (A) and $p$-values (B) for the null hypothesis of no change
	in lipid level in ChAc w.r.t Control in the CN region (H0: $\beta_{3g}=0$)
	for each of the 34 lipid groups.  Top and bottom \emph{plots} show results
	based on the group-level and species-level data, respectively.  The smaller,
	$34 \times 30$-sized, group-level data matrix was made from the larger, $593
	\times 30$-sized species-level data matrix by summing lipid level across all
	species withing each lipid group.  This means $n_g\times$ more observations
	on each lipid group $g$ in the species-level data, where $n_g$ is the number
	of species within group $g$.  We see that the species-level data affords
	more statistical power: larger $|t|$-values and smaller $p$-values, hence
	this data set was used for final inferences.  The top and bottom \emph{rows}
	within each plot show results from fitting a fixed and a mixed model,
	respectively.  The mixed model---used for our final inferences---contains a
	random effects term that models subject-to-subject variation in lipid level,
	otherwise it is identical to the fixed model.  We see higher $|t|$-values
	and lower $p$-values for the fixed model in the species-level data
	indicating a bias from failing to account for subject-to-subject variation.
	The mixed model corrects for that bias.  Taken together, these findings
	justify application of the mixed model to the species-level data, which we
	done in the present study.
}
\label{fig:t-value-heatmap}
\end{figure}

%\begin{figure}[p]
%	\includegraphics[width=0.5\textwidth]{../../notebooks/2023-01-04-side-chain/named-figure/level-sidechain_len-SM.pdf}
%	\includegraphics[width=0.5\textwidth]{../../notebooks/2023-01-04-side-chain/named-figure/level-sidechain_len-PC.pdf}
%\caption{
%	Distribution of side chain length in control and ChAc samples
%}
%\label{fig:sidechain-SM}
%\end{figure}

\begin{figure}[p]
\begin{center}
	\includegraphics[width=0.4\textwidth]{../../notebooks/2022-09-04-mixed-models/named-figure/pvalues-PMI-effect.pdf}
\end{center}
	\caption[Significance of the effect of PMI on lipid levels]{
		Significance of the effect of PMI on lipid levels.  Note $-\log_{10}$ scaled $x$
		axis.
}
\label{fig:PMI-pvals}
\end{figure}

\begin{figure}[p]
\begin{center}
	\includegraphics[width=0.8\textwidth]{../../notebooks/2022-09-04-mixed-models/named-figure/PMI-effect-on-estimate-pval.pdf}
\end{center}
	\caption[Comparing key results without PMI correction to those with PMI correction]{
		Comparing key results without PMI correction ($x$ axes) to those with PMI
		correction ($y$ axes).  Left: estimates and errors
		($\hat{\beta}_{\mathrm{Dx[T.ChAc]}:\mathrm{Region[}r\mathrm{]}} \pm
		\mathrm{SE}$) of region $r$-specific change of lipid level in ChAc.
		Estimates on standardized scale.
		Right: corresponding $p$-values on $-\log_{10}$ scale.
}
\label{fig:PMI-effect-on-estimates}
\end{figure}

\begin{figure}[p]
\begin{center}
	\includegraphics[width=1.0\textwidth]{../../notebooks/2022-09-04-mixed-models/named-figure/pvalues-corrected-wo-PMI.pdf}
\end{center}
	\caption[$p$-values for region-specific change in ChAc with and without correction for PMI]{
		$p$-values for region-specific change in ChAc with and without correction
		for PMI (+ PMI, - PMI, respectively).  $p$-values reaching significance
		after Benjamini-Hochberg correction are shown as filled symbols.
}
\label{fig:pvalues-corrected-wo-PMI}
\end{figure}


\begin{figure}[p]
\begin{center}
	\includegraphics[width=1.0\textwidth]{../../notebooks/2022-09-04-mixed-models/named-figure/line-ensemble.pdf}
\end{center}
	\caption[Significant lipid changes in individual control and ChAc patients I.]{
		Lipid groups with significantly increased levels in the CN and putamen of
		individual control and ChAc patients. Empty circles show actual
		standardized lipid
		levels; the lines connecting circles for individual subjects are only
		included to enhance visual interpretation. Small circles and thin lines
		indicate standardized levels of the individual lipid species. Large circles
		and thick lines show the mean levels across species for each lipid group and
		subject.
}
\label{fig:individual-changes}
\end{figure}


\begin{figure}[p]
\begin{center}
	\includegraphics[width=1.0\textwidth]{../../notebooks/2022-09-04-mixed-models/named-figure/line-ensemble-green-purple.pdf}
\end{center}
	\caption[Significant lipid changes in individual control and ChAc patients II.]{
		Lipid groups with significantly increased levels in the CN and putamen of
		individual control and ChAc patients. As Fig.~S\ref{fig:individual-changes} but only mean standardized levels across
		individual lipid species are shown and subjects have been re-colored.
		Although individual subjects show considerable variation, the
		brain-regional distribution of metabolite levels is clearly different in ChAc subjects
		relative to Control.
}
\label{fig:individual-changes-simple}
\end{figure}

\begin{figure}[p]
	\includegraphics[width=0.75\textwidth]{../../notebooks/2023-01-04-side-chain/named-figure/level-1_sidechain_len-BMP.pdf}
	\caption[BMP levels across various side chain lengths and degrees of unsaturation]{
		Standardized BMP levels across various side chain lengths and degrees of unsaturation.
		$y$ axes: BMP level. $x$ axes: side chain length. Rows: 
		degrees of unsaturation, labelled as $\mathrm{du} = 0,...$). Columns:
		brain regions.
		Note that for each side chain length at each
		graph we have six Control points and four ChAc points (blue and orange $+$
		symbols, respectively).  Colored straight solid lines: ordinary
		least square fitted curves.  Colored shades: confidence intervals.
}
\label{fig:sidechain-BMP}
\end{figure}


\begin{figure}[p]
	\includegraphics[width=0.75\textwidth]{../../notebooks/2023-01-04-side-chain/named-figure/level-1_sidechain_len-LPS.pdf}
	\caption[LPS levels across various side chain lengths and degrees of unsaturation]{
		Standardized LPS levels across various side chain lengths and degrees of unsaturation.
		See Fig.~S\ref{fig:sidechain-BMP} for details.
}
\label{fig:sidechain-LPS}
\end{figure}


\begin{figure}[p]
	\includegraphics[width=0.75\textwidth]{../../notebooks/2023-01-04-side-chain/named-figure/level-1_sidechain_len-PCe.pdf}
	\caption[PCe levels across various side chain lengths and degrees of unsaturation]{
		Standardized PCe levels across various side chain lengths and degrees of unsaturation.
		See Fig.~S\ref{fig:sidechain-BMP} for details.
}
\label{fig:sidechain-PCe}
\end{figure}


\begin{figure}[p]
	\includegraphics[width=0.75\textwidth]{../../notebooks/2023-01-04-side-chain/named-figure/level-2_sidechain_len-Sulf.pdf}
	\caption[Sulf levels across various side chain lengths and degrees of unsaturation]{
		Standardized Sulf levels across various side chain lengths and degrees of unsaturation.
		$y$ axes: BMP level. $x$ axes: side chain length. Rows: 
		degrees of unsaturation, labelled as $\mathrm{du} = (0,0), ...$). Columns:
		brain regions.
	While BMP, LPS and PCe
	(Fig.~S\ref{fig:sidechain-BMP}-S\ref{fig:sidechain-PCe}) have only one side
	chain, Sulf has two.  Hence, we have, e.g, $\mathrm{du} = (1,0)$, which means
	degree of unsaturation of 1 for chain 1 and that of 0 for chain 2.  The
	length displayed on the $x$ axis is that of side chain 2, while side chain 1
	is invariably of length 18.
}
\label{fig:sidechain-Sulf}
\end{figure}


\end{document}
